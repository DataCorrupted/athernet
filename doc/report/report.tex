%% Computer Networks Project Report Latex file.
%% Completed By Yuyang Rong(rongyy@shanghaitech.edu.cn) and 
%% Jianxiong Cai(caijx@shanghaitech.edu.cn)
%%
%% To edit this file, please use indentions with tab size of 2.
%%

\documentclass[conference,compsoc]{IEEEtran}
\usepackage{cite}
\usepackage{listings}
\usepackage{mathtools}
\usepackage{amsmath,amsthm,amssymb,amsfonts}


\begin{document}
\title{
	Computer Networks Course Report
}
% author names and affiliations
% use a multiple column layout for up to three different
% affiliations
\author{
	\IEEEauthorblockN{Yuyang Rong}
	\IEEEauthorblockA{
		School of Information Science and Technology \\
		ShanghaiTech University \\
		Student ID: 69850764 \\
	}
\and
	\IEEEauthorblockN{Jianxiong Cai}
	\IEEEauthorblockA{
		School of Information Science and Technology \\
		ShanghaiTech University \\
		Student ID: 67771603 \\
	}
}

\maketitle

% As a general rule, do not put math, special symbols or citations
% in the abstract
\begin{abstract}
A quick brown fox jumps over a lazy dog.
\end{abstract}

%%%%%%%%%%%%%%%%%%%%%%%%%%%%%%%%%%%%%%%%%%%%%%%%%%%%%%%%%%%%%%%%%%%%%
%%
%% Project 1: physical layer
%%
%%%%%%%%%%%%%%%%%%%%%%%%%%%%%%%%%%%%%%%%%%%%%%%%%%%%%%%%%%%%%%%%%%%%%
\section{Project 1: physical layer}
	
	%TODO
	\subsection{Sound card and DataLine}
		A quick brown fox jumps over a lazy dog.
	
	%TODO
	\subsection{Java experience: OOP}
		A quick brown fox jumps over a lazy dog.

%%%%%%%%%%%%%%%%%%%%%%%%%%%%%%%%%%%%%%%%%%%%%%%%%%%%%%%%%%%%%%%%%%%%%
%%
%% Project 2: mac layer
%%
%%%%%%%%%%%%%%%%%%%%%%%%%%%%%%%%%%%%%%%%%%%%%%%%%%%%%%%%%%%%%%%%%%%%%
\section{Project 2: mac layer}
	
	%TODO
	\subsection{Design}
		The mac layer has 3 main components, corresponding to 3 independent thread. The thread detail will be tackled in \ref{Thread pool}. Here, we will only talk about what each component do.
		\subsubsection{Request}
			We can not allow packet to be send 
		\subsubsection{Send}
		\subsubsection{Receive}

	%TODO
	\subsection{Thread pool}\label{Thread pool}
		Mention we have two threads and how they help with each other.
		How do we manage to prevent thread bugs.
	
	%TODO
	\subsection{Sequential arriving data}
	
	%TODO
	\subsection{Mac Packet}
		Ernest please fill this part.
	
	%TOCHECK
	\subsection{Signal Detection}
		Whenever sending a frame, the sender must make sure that the channel is clean and is good for transmittion.
		The process is somewhat similar to CSMA/CD, yet there is no detecting while sending.
		\par
		The sending process will start only when the moving average power of the channel is low enough. The avg power should satisfy:
		\begin{equation}\begin{aligned}
		avg_k \\
		& = 
			\frac{win\_size-1}{win\_size}avg_{k-1} + 
			\frac{1}{win\_size} * power_k \\
		& > thr
		\end{aligned}\end{equation}
		where in our implementation, we have:
		\begin{equation*}\begin{aligned} 
			& win\_size = 24 \\
			& thr = 0.3 
		\end{aligned}\end{equation*}
		\par
		If the power is	greater than the threshold we set, we will wait sometime, by default we wait 10ms.
		\par 
		We didn't add collection detection in out implemtation due to one design flaw. We split the physical layer to two independent part: sender and receiver. The fact that these two components cannot communate leads to the hardship when writing collection detection. We hold one assumption while doing this project, that once the transmittion is started, no one will interupted because other senders should be listening the channel and hold off sending only when the average power goes down.
	%TODO
	\subsection{Java experience: reference and package}
		Peter please fill this part.

%%%%%%%%%%%%%%%%%%%%%%%%%%%%%%%%%%%%%%%%%%%%%%%%%%%%%%%%%%%%%%%%%%%%%
%%
%% Project 3: gateway
%%
%%%%%%%%%%%%%%%%%%%%%%%%%%%%%%%%%%%%%%%%%%%%%%%%%%%%%%%%%%%%%%%%%%%%%
\section{Project 3: gateway}

%TODO: insert a graph to illistrate the overall layout in this part
	
	%TODO
	\subsection{JNI}
		Peter please fill this part.
	
	%TODO
	\subsection{Pipe}
		Peter please fill this part.
		
		%TODO: maybe nat_pack should be mentioned here? 
	
	%TODO
	\subsection{C++ experience: Socket Programming}
		\paragraph{\textbf{gateway on the c++ side}}
		The c++ need to handle all Internet Communication and provide I/O service for the java gateway agent. In other words, the c++ gateway need to send out IP packets whenever the java gateway request sending, and give received packet from Internet to Java Interface (with some pre-pocessing).
		
		
		\paragraph{\textbf{TCP, UDP and ICMP}}
		For TCP and UDP, we are calling the interfaces provided by linux keneral interface, inlcuding \emph{socket}, \emph{binding}, \emph{connect}, \emph{sendto} and \emph{recefrom}. For ICMP, we are using the ICMP provided by Boost Libraries. 
		\paragraph{\textbf{TCP Receiving Buffer}}
		 One tricky part here is that TCP is streaming rather than transmitting packets, which results in that one frame may split into two frames when sending out. In order to solve the problem, for some application scenarioes, we use first several bytes to encode the length of the frame on sending and receiver store the current frame to buffer if it need to wait for the next frame.
		

%%%%%%%%%%%%%%%%%%%%%%%%%%%%%%%%%%%%%%%%%%%%%%%%%%%%%%%%%%%%%%%%%%%%%
%%
%% Project 4: FTP application
%%
%%%%%%%%%%%%%%%%%%%%%%%%%%%%%%%%%%%%%%%%%%%%%%%%%%%%%%%%%%%%%%%%%%%%%
\section{Project 4: FTP application}
	\subsection{Overview}	
	The FTP connection is handled mainly at the gateway side. The node A (athernet FTP client) is connected to node B through athernet. The node B (FTP gateway) is connected to FTP Server via Internet.
	\subsection{Passive Mode}
	For \emph{LIST} and \emph{RETR} command, passive mode need to be supported. The gateway parses every reply from FTP server before redirecting that to java client. Once the gateway received a reply starting with code 227, the reply from server with expected passive ports, the gateway records it as \emph{data\_port}. Then, if the gateway received a sending request of \emph{LIST} or \emph{RETR}, it immediately start a new TCP connection to the server with \emph{data\_port}.

%%%%%%%%%%%%%%%%%%%%%%%%%%%%%%%%%%%%%%%%%%%%%%%%%%%%%%%%%%%%%%%%%%%%%
%%
%% Course and Homework
%%
%%%%%%%%%%%%%%%%%%%%%%%%%%%%%%%%%%%%%%%%%%%%%%%%%%%%%%%%%%%%%%%%%%%%%
\section{Course \& Homework}
	
	%TODO
	\subsection{Course}
	
	%TODO
	\subsection{Homework}

%%%%%%%%%%%%%%%%%%%%%%%%%%%%%%%%%%%%%%%%%%%%%%%%%%%%%%%%%%%%%%%%%%%%%
%%
%% Acknowledgement
%%
%%%%%%%%%%%%%%%%%%%%%%%%%%%%%%%%%%%%%%%%%%%%%%%%%%%%%%%%%%%%%%%%%%%%%
\section*{Acknowledgment}
	%TODO

\bibliographystyle{IEEEtran}
\bibliography{report}


% that's all folks
\end{document}



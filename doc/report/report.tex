%% Computer Networks Project Report Latex file.
%% Completed By Yuyang Rong(rongyy@shanghaitech.edu.cn) and 
%% Jianxiong Cai(caijx@shanghaitech.edu.cn)
%%
%% To edit this file, please use indentions with tab size of 2.
%%

\documentclass[conference,compsoc]{IEEEtran}
\usepackage{cite}
\usepackage{listings}
\usepackage{mathtools}
\usepackage{amsmath,amsthm,amssymb,amsfonts}


\begin{document}
\title{
	Computer Networks Course Report
}
% author names and affiliations
% use a multiple column layout for up to three different
% affiliations
\author{
	\IEEEauthorblockN{Yuyang Rong}
	\IEEEauthorblockA{
		School of Information Science and Technology \\
		ShanghaiTech University \\
		Student ID: 69850764 \\
	}
\and
	\IEEEauthorblockN{Jianxiong Cai}
	\IEEEauthorblockA{
		School of Information Science and Technology \\
		ShanghaiTech University \\
		Student ID: 67771603 \\
	}
}

\maketitle

% As a general rule, do not put math, special symbols or citations
% in the abstract
\begin{abstract}
A quick brown fox jumps over a lazy dog.
\end{abstract}

%%%%%%%%%%%%%%%%%%%%%%%%%%%%%%%%%%%%%%%%%%%%%%%%%%%%%%%%%%%%%%%%%%%%%
%%
%% Project 1: physical layer
%%
%%%%%%%%%%%%%%%%%%%%%%%%%%%%%%%%%%%%%%%%%%%%%%%%%%%%%%%%%%%%%%%%%%%%%
\section{Project 1: physical layer}
	
	%TODO
	\subsection{Sound card and DataLine}
		A quick brown fox jumps over a lazy dog.
	
	%TODO
	\subsection{Java experience: OOP}
		A quick brown fox jumps over a lazy dog.

%%%%%%%%%%%%%%%%%%%%%%%%%%%%%%%%%%%%%%%%%%%%%%%%%%%%%%%%%%%%%%%%%%%%%
%%
%% Project 2: mac layer
%%
%%%%%%%%%%%%%%%%%%%%%%%%%%%%%%%%%%%%%%%%%%%%%%%%%%%%%%%%%%%%%%%%%%%%%
\section{Project 2: mac layer}
	
	%TODO
	\subsection{Design}
		The mac layer has 3 main components, corresponding to 3 independent thread. The thread detail will be tackled in \ref{Thread pool}. Here, we will only talk about what each component do.
		\subsubsection{Request}
		\subsubsection{Send}
		\subsubsection{Receive}

	%TODO
	\subsection{Thread pool}\label{Thread pool}
		Mention we have two threads and how they help with each other.
		How do we manage to prevent thread bugs.
	
	%TODO
	\subsection{Sequential arriving data}
	
	%TODO
	\subsection{Mac Packet}
		Ernest please fill this part.
	
	%TOCHECK
	\subsection{Signal Detection}
		Whenever sending a frame, the sender must make sure that the channel is clean and is good for transmittion.
		The process is somewhat similar to CSMA/CD, yet there is no detecting while sending.
		\par
		The sending process will start only when the moving average power of the channel is low enough. The avg power should satisfy:
		\begin{equation}\begin{aligned}
		avg_k \\
		& = 
			\frac{win\_size-1}{win\_size}avg_{k-1} + 
			\frac{1}{win\_size} * power_k \\
		& > thr
		\end{aligned}\end{equation}
		where in our implementation, we have:
		\begin{equation*}\begin{aligned} 
			& win\_size = 24 \\
			& thr = 0.3 
		\end{aligned}\end{equation*}
		\par
		If the power is	greater than the threshold we set, we will wait sometime, by default we wait 10ms.
		\par 
		We didn't add collection detection in out implemtation due to one design flaw. We split the physical layer to two independent part: sender and receiver. The fact that these two components cannot communate leads to the hardship when writing collection detection. We hold one assumption while doing this project, that once the transmittion is started, no one will interupted because other senders should be listening the channel and hold off sending only when the average power goes down.
	%TODO
	\subsection{Java experience: reference and package}
		Peter please fill this part.

%%%%%%%%%%%%%%%%%%%%%%%%%%%%%%%%%%%%%%%%%%%%%%%%%%%%%%%%%%%%%%%%%%%%%
%%
%% Project 3: gateway
%%
%%%%%%%%%%%%%%%%%%%%%%%%%%%%%%%%%%%%%%%%%%%%%%%%%%%%%%%%%%%%%%%%%%%%%
\section{Project 3: gateway}
	
	%TODO
	\subsection{JNI}
		Peter please fill this part.
	
	%TODO
	\subsection{Pipe}
		Peter please fill this part.
	
	%TODO
	\subsection{C++ experience: Boost}
		Ernest please fill this part.

%%%%%%%%%%%%%%%%%%%%%%%%%%%%%%%%%%%%%%%%%%%%%%%%%%%%%%%%%%%%%%%%%%%%%
%%
%% Project 4: FTP application
%%
%%%%%%%%%%%%%%%%%%%%%%%%%%%%%%%%%%%%%%%%%%%%%%%%%%%%%%%%%%%%%%%%%%%%%
\section{Project 4: FTP application}
	%TODO
	Ernest please fill this part.

%%%%%%%%%%%%%%%%%%%%%%%%%%%%%%%%%%%%%%%%%%%%%%%%%%%%%%%%%%%%%%%%%%%%%
%%
%% Course and Homework
%%
%%%%%%%%%%%%%%%%%%%%%%%%%%%%%%%%%%%%%%%%%%%%%%%%%%%%%%%%%%%%%%%%%%%%%
\section{Course \& Homework}
	
	%TODO
	\subsection{Course}
	
	%TODO
	\subsection{Homework}

%%%%%%%%%%%%%%%%%%%%%%%%%%%%%%%%%%%%%%%%%%%%%%%%%%%%%%%%%%%%%%%%%%%%%
%%
%% Acknowledgement
%%
%%%%%%%%%%%%%%%%%%%%%%%%%%%%%%%%%%%%%%%%%%%%%%%%%%%%%%%%%%%%%%%%%%%%%
\section*{Acknowledgment}
	%TODO

\bibliographystyle{IEEEtran}
\bibliography{report}


% that's all folks
\end{document}


